\usepackage[dvipdfmx]{graphicx} % images.
\usepackage{tikz} % TikZ ist kein Zeichenprogramm.
\usepackage{tikz-cd} % TikZ + commutative diagrams.

\usepackage{amsmath,amssymb,amsthm} % math symbols, theorem-like, etc.
\usepackage{bm} % access bold symbols in maths mode.
\usepackage{mathrsfs} % Ralph Smith's Formal Script, \mathscr{}.
%\usepackage[euler-digits,euler-hat-accent]{eulervm} % Euler fonts.
%\usepackage{newpxtext,newpxmath} % Palatino fonts.
%\usepackage{newtxtext,newtxmath} % Times fonts.
\usepackage{mathtools} % more math symbols, arrows, etc.
\usepackage{halloweenmath} % 👻
\usepackage{physics} % macros supporting the mathematics of physics.
\usepackage{lipsum} % easy access to the lorem ipsum and other dummy texts.
\usepackage{etoolbox} % e-TeX tools for LaTeX.

\usepackage{mdframed} % frames, boxes, etc.
\usepackage{tcolorbox} % colored boxes.
\usepackage{ascmac} % boxes and picture macros with Japanese vertical writing support

\usepackage{enumitem} % replaces old enumerate.
\renewcommand{\labelenumi}{(\theenumi)}
\usepackage{xcolor} % driver-independent color extensions 
\usepackage{hyperref} % hyperlinks, navigations.
\usepackage{cleveref} % clever references.
\usepackage{fancyhdr} % fancy headers and footers.
\usepackage{lastpage} % reference last page for page N of M type footers.
\usepackage{setspace} % line spacing.

\usepackage[utf8]{inputenc}
\usepackage[autolanguage]{numprint}

\usepackage{pxrubrica} % Japanese ruby.
