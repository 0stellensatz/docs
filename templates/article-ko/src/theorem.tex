\newcommand{\termTheorem}{정리\rmfamily}
\newcommand{\termAxiom}{공리\rmfamily}
\newcommand{\termProposition}{명제\rmfamily}
\newcommand{\termLemma}{보조정리\rmfamily}
\newcommand{\termConjecture}{추측\rmfamily}
\newcommand{\termCorollary}{따름정리\rmfamily}
\newcommand{\termDefinition}{정의\rmfamily}
\newcommand{\termProblem}{문제\rmfamily}
\newcommand{\termSolution}{풀이\rmfamily}
\newcommand{\termExample}{예\rmfamily}
\newcommand{\termRemark}{\normalfont 주\rmfamily}
\newcommand{\termNote}{\normalfont 노트\rmfamily}
\newcommand{\termProof}{\normalfont 증명\rmfamily}
\newcommand{\termNumbered}{\ignorespaces}

\newtheoremstyle{cjkplain}% name
{}% space above
{}% space below
{\upshape}% body font
{}% indent amount
{\bfseries}% theorem head font
{.}% punctuation after theorem head
{.5em}% space after theorem head
{}% theorem head spec (can be left empty, meaning `normal')

\makeatletter
\def\cleartheorem#1{%
    \expandafter\let\csname#1\endcsname\relax
    \expandafter\let\csname c@#1\endcsname\relax
}
\makeatother

\theoremstyle{cjkplain}

\newtheorem{theorem}{\termTheorem}[section]
\newtheorem{conjecture}[theorem]{\termConjecture}
\newtheorem{lemma}[theorem]{\termLemma}
\newtheorem{proposition}[theorem]{\termProposition}
\newtheorem{corollary}[theorem]{\termCorollary}
\newtheorem{problem}[theorem]{\termProblem}

\theoremstyle{definition}

\newtheorem{definition}[theorem]{\termDefinition}
\newtheorem{example}[theorem]{\termExample}

\newtheorem{numbered}[theorem]{\termNumbered}

\theoremstyle{remark}

\newtheorem*{note}{\termNote}
\newtheorem*{remark}{\termRemark}
\newtheorem*{solution}{\termSolution}

\renewcommand\qedsymbol{$\square$}
