\newcommand{\termTheorem}{Theorem}
\newcommand{\termAxiom}{Axiom}
\newcommand{\termProposition}{Proposition}
\newcommand{\termLemma}{Lemma}
\newcommand{\termConjecture}{Conjecture}
\newcommand{\termCorollary}{Corollary}
\newcommand{\termDefinition}{Definition}
\newcommand{\termProblem}{Problem}
\newcommand{\termSolution}{Solution}
\newcommand{\termExample}{Example}
\newcommand{\termRemark}{Remark}
\newcommand{\termNote}{Note}
\newcommand{\termProof}{Proof}
\newcommand{\termNumbered}{\ignorespaces}

\newtheoremstyle{latinplain}% name
{16pt}% space above
{16pt}% space below
{}% body font
{}% indent amount
{\bfseries}% theorem head font
{.}% punctuation after theorem head
{.5em}% space after theorem head
{}% theorem head spec (can be left empty, meaning `normal')

\makeatletter
\def\cleartheorem#1{%
    \expandafter\let\csname#1\endcsname\relax
    \expandafter\let\csname c@#1\endcsname\relax
}
\makeatother

% e.g. re-declaring an existing theorem-like environment:
%
% \cleartheorem{problem}
% \theoremstyle{cjkplain}
% \newtheorem{problem}{\termProblem}[subsection]

\theoremstyle{latinplain}

\newtheorem{theorem}{\termTheorem}[section]
\newtheorem{conjecture}[theorem]{\termConjecture}
\newtheorem{lemma}[theorem]{\termLemma}
\newtheorem{proposition}[theorem]{\termProposition}
\newtheorem{corollary}[theorem]{\termCorollary}
\newtheorem{problem}{\termProblem}

\theoremstyle{definition}

\newtheorem{definition}[theorem]{\termDefinition}
\newtheorem{example}[theorem]{\termExample}

\newtheorem{numbered}[theorem]{\termNumbered}

\theoremstyle{remark}

\newtheorem*{note}{\termNote}
\newtheorem*{remark}{\termRemark}
\newtheorem*{solution}{\termSolution}

\renewcommand\qedsymbol{$\square$}
%\renewcommand\qedsymbol{$\blacksquare$}
